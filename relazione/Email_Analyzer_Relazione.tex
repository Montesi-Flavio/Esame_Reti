\documentclass{article}

% UTF-8 encoding support
\usepackage[utf8]{inputenc}
\usepackage[T1]{fontenc}

% Language setting
\usepackage[italian]{babel}

% Set page size and margins
\usepackage[a4paper,top=2cm,bottom=2cm,left=2cm,right=2cm,marginparwidth=1.5cm]{geometry}

% Useful packages
\usepackage{amsmath}
\usepackage{graphicx}
\usepackage{listings}
\usepackage{xcolor}
\usepackage{fancyhdr}

% Listings configuration
\lstset{
    basicstyle=\ttfamily\footnotesize,
    backgroundcolor=\color{gray!10},
    frame=single,
    breaklines=true,
    captionpos=b,
    numbers=left,
    numberstyle=\tiny\color{gray},
    keywordstyle=\color{blue},
    commentstyle=\color{green!60!black},
    stringstyle=\color{red},    showstringspaces=false
}

% Hyperref deve essere caricato per ultimo
\usepackage[colorlinks=true, allcolors=blue]{hyperref}

\title{Email Analyzer\\Sistema di Analisi di Sicurezza Multi-Servizio\\per Email IMAP}
\author{Montesi Flavio}
\date{Giugno 2025}

% Header and footer
\pagestyle{fancy}
\fancyhf{}
\fancyhead[L]{Email Analyzer - Sistema Multi-Servizio}
\fancyhead[R]{Montesi Flavio}
\fancyfoot[C]{\thepage}

\begin{document}

\maketitle 
\tableofcontents
\newpage

\section{Introduzione}

L'applicazione "Email Analyzer" è uno strumento avanzato progettato per analizzare le email scaricate da un server IMAP con un approccio multi-servizio alla sicurezza informatica. Questo sistema fornisce un'analisi completa e dettagliata di vari componenti delle email, inclusi header, link, allegati e verifiche di autenticazione.

Il progetto è stato sviluppato come evoluzione di un sistema inizialmente basato esclusivamente su VirusTotal, trasformandolo in una piattaforma robusta che integra oltre 8 servizi di sicurezza diversi, garantendo così una copertura completa delle minacce digitali e riducendo la dipendenza da un singolo fornitore.

\section{Architettura del Sistema}

\subsection{Panoramica Generale}

Il sistema Email Analyzer è strutturato secondo un'architettura modulare che separa chiaramente le responsabilità:

\begin{itemize}
    \item \textbf{Core Application} (`app.py`): Entry point principale con gestione parametri CLI
    \item \textbf{Email Processing} (`email\_core.py`): Gestione connessioni IMAP e elaborazione email
    \item \textbf{Analysis Modules} (`analyzers/`): Moduli specializzati per diversi tipi di analisi
    \item \textbf{Security Connectors} (`connectors.py`): Interfacce con servizi di sicurezza esterni
    \item \textbf{Output Generation} (`output/`): Generazione report JSON e HTML
    \item \textbf{Configuration} (`config.py`): Configurazione centralizzata e chiavi API
\end{itemize}

\subsection{Moduli di Analisi Specializzati}

Il sistema include quattro moduli di analisi specializzati:

\begin{enumerate}
    \item \textbf{Header Analyzer}: Analisi completa degli header email con categorizzazione automatica
    \item \textbf{Link Analyzer}: Estrazione e verifica sicurezza di tutti i link presenti
    \item \textbf{Attachment Analyzer}: Analisi hash degli allegati e rilevamento malware
    \item \textbf{DMARC Analyzer}: Verifica protocolli di autenticazione DKIM, SPF e DMARC
\end{enumerate}

\section{Funzionalità Principali}

\subsection{Download e Gestione Email}

\subsubsection{IMAP (Internet Message Access Protocol)}
\textbf{Definizione:} Protocollo standard per l'accesso alle email memorizzate su un server di posta. A differenza di POP3, IMAP mantiene i messaggi sul server permettendo l'accesso da più dispositivi.

\textbf{Caratteristiche principali:}
\begin{itemize}
    \item Sincronizzazione bidirezionale tra client e server
    \item Supporto per cartelle multiple e gerarchiche
    \item Accesso parziale ai messaggi (solo header, solo corpo, ecc.)
    \item Gestione degli stati dei messaggi (letto, non letto, eliminato)
\end{itemize}

\subsubsection{Implementazione nel Sistema}
Il sistema si connette a server IMAP utilizzando SSL/TLS e scarica le email mantenendole nel loro formato originale (.eml). La gestione include:

\begin{itemize}
    \item Connessione sicura SSL al server IMAP
    \item Autenticazione con credenziali utente
    \item Download selettivo per mailbox specifica
    \item Preservazione formato originale email
    \item Gestione errori di connessione e timeout
\end{itemize}

\subsection{Analisi degli Header}

\subsubsection{Header Email}
\textbf{Definizione:} Metadati strutturati che precedono il corpo di un'email, contenenti informazioni tecniche sul percorso, autenticazione e caratteristiche del messaggio.

\textbf{Categorie principali:}
\begin{itemize}
    \item \textbf{Header di Routing:} Tracciano il percorso del messaggio (Received, Return-Path)
    \item \textbf{Header di Identificazione:} Identificano mittente e destinatario (From, To, Message-ID)
    \item \textbf{Header di Autenticazione:} Contengono informazioni di verifica (DKIM-Signature, Authentication-Results)
    \item \textbf{Header di Sicurezza:} Risultati di analisi antispam e antimalware (X-Spam-Status, X-Virus-Scan)
\end{itemize}

\subsubsection{Implementazione Completa di Categorizzazione}

Il sistema implementa un'analisi categorizzata degli header email con le seguenti caratteristiche:

\begin{itemize}
    \item \textbf{Categorizzazione Automatica}: Header suddivisi per tipologia (Autenticazione, Sicurezza, Routing, Tecnici)
    \item \textbf{Visualizzazione Interattiva}: Sistema di espansione/compressione per header lunghi
    \item \textbf{Analisi di Sicurezza}: Estrazione IP mittente e verifica blacklist
    \item \textbf{Investigazione Avanzata}: Controllo reputazione IP su servizi multipli
\end{itemize}

\begin{lstlisting}[language=Python, caption=Esempio categorizzazione header]
# Categorie header automatiche
auth_headers = [
    'dkim-signature', 'arc-seal', 'arc-authentication-results', 
    'authentication-results', 'received-spf'
]

security_headers = [
    'x-spam-status', 'x-rspamd-server', 'x-cnfs-analysis',
    'x-spam-score', 'x-spam-level'
]

routing_headers = [
    'received', 'return-path', 'delivered-to', 'x-received',
    'x-originating-ip'
]

def categorize_header(header_name):
    """Categorizza un header basandosi sul nome"""
    header_lower = header_name.lower()
      if header_lower in auth_headers:
        return "Authentication & Security"
    elif header_lower in security_headers:
        return "Security & Spam"
    elif header_lower in routing_headers:
        return "Routing & Delivery"
    else:
        return "Other"
\end{lstlisting}

\subsection{Analisi dei Link}

\subsubsection{Phishing}
\textbf{Definizione:} Tecnica di ingegneria sociale che utilizza comunicazioni elettroniche fraudolente per rubare informazioni sensibili o credenziali, facendosi passare per entità affidabili.

\textbf{Caratteristiche tipiche:}
\begin{itemize}
    \item URL contraffatti che imitano siti legittimi
    \item Richieste urgenti di aggiornamento credenziali
    \item Loghi e grafica copiati da organizzazioni reali
    \item Domini simili a quelli originali (typosquatting)
\end{itemize}

\subsubsection{Sistema di Analisi Comprehensive}

L'analisi dei link implementa un approccio multi-servizio con le seguenti funzionalità:

\begin{itemize}
    \item \textbf{Estrazione Intelligente}: Rilevamento URL da contenuto HTML e testo
    \item \textbf{Normalizzazione}: Standardizzazione URL per evitare duplicati
    \item \textbf{Batch Processing}: Elaborazione in lotti per rispettare rate limits
    \item \textbf{Caching Avanzato}: Sistema di cache per ridurre chiamate API
    \item \textbf{Risk Scoring}: Punteggio di rischio basato su analisi multi-servizio
\end{itemize}

\subsection{Analisi degli Allegati}

\subsubsection{Hash Crittografici}
\textbf{Definizione:} Funzioni matematiche che convertono dati di qualsiasi dimensione in una stringa di caratteri di lunghezza fissa, utilizzate per verificare l'integrità e identificare univocamente i file.

\textbf{Algoritmi utilizzati:}
\begin{itemize}
    \item \textbf{MD5:} 128 bit, veloce ma vulnerabile a collisioni
    \item \textbf{SHA1:} 160 bit, deprecato per uso crittografico
    \item \textbf{SHA256:} 256 bit, standard attuale per sicurezza
\end{itemize}

\subsubsection{Malware via Email}
\textbf{Definizione:} Software dannoso distribuito attraverso allegati email o link che scaricano payload malevoli.

\textbf{Tipologie comuni:}
\begin{itemize}
    \item \textbf{Trojan:} Software che si nasconde come applicazione legittima
    \item \textbf{Ransomware:} Crittografa i file richiedendo un riscatto
    \item \textbf{Stealer:} Ruba credenziali e informazioni sensibili
    \item \textbf{Botnet Agent:} Trasforma il sistema in parte di una rete botnet
\end{itemize}

\subsubsection{Implementazione nel Sistema}

Il sistema calcola hash multipli (MD5, SHA1, SHA256) per ogni allegato e li verifica contro database di malware:

\begin{itemize}
    \item Calcolo hash multipli per maggiore sicurezza
    \item Verifica su VirusTotal e MalwareBazaar
    \item Identificazione famiglia malware
    \item Sistema di quarantena automatica per file sospetti
\end{itemize}

\subsection{Verifica Autenticazione Email}

\subsubsection{SPF (Sender Policy Framework)}
\textbf{Definizione:} Meccanismo di autenticazione email che consente ai proprietari di domini di specificare quali server di posta sono autorizzati a inviare email per conto del loro dominio.

\textbf{Funzionamento:}
\begin{itemize}
    \item Pubblicazione di record DNS TXT contenenti le policy SPF
    \item Verifica da parte del server ricevente dell'IP mittente contro il record SPF
    \item Risultati possibili: Pass, Fail, SoftFail, Neutral, None, TempError, PermError
\end{itemize}

\textbf{Esempio record SPF:}
\begin{lstlisting}[caption=Esempio record SPF]
v=spf1 ip4:192.168.1.0/24 include:_spf.google.com -all
\end{lstlisting}

\subsubsection{DKIM (DomainKeys Identified Mail)}
\textbf{Definizione:} Sistema di autenticazione che utilizza crittografia a chiave pubblica per verificare che un'email non sia stata alterata durante il transito e che provenga effettivamente dal dominio dichiarato.

\textbf{Componenti chiave:}
\begin{itemize}
    \item \textbf{Chiave privata:} Utilizzata dal server mittente per firmare l'email
    \item \textbf{Chiave pubblica:} Pubblicata nei record DNS per la verifica
    \item \textbf{Selettore:} Identificatore che specifica quale chiave utilizzare
    \item \textbf{Firma digitale:} Hash crittografico di header e corpo specifici
\end{itemize}

\textbf{Struttura firma DKIM:}
\begin{lstlisting}[caption=Struttura firma DKIM]
DKIM-Signature: v=1; a=rsa-sha256; d=example.com; s=selector1;
    h=from:to:subject:date; bh=hash_corpo; b=firma_crittografata
\end{lstlisting}

\subsubsection{DMARC (Domain-based Message Authentication, Reporting \& Conformance)}
\textbf{Definizione:} Policy di autenticazione che si basa su SPF e DKIM per determinare l'autenticità di un'email e specificare le azioni da intraprendere per i messaggi che falliscono l'autenticazione.

\textbf{Policy DMARC:}
\begin{itemize}
    \item \textbf{none:} Solo monitoraggio, nessuna azione
    \item \textbf{quarantine:} Messaggio considerato sospetto (spam folder)
    \item \textbf{reject:} Messaggio rifiutato completamente
\end{itemize}

\textbf{Allineamento:}
\begin{itemize}
    \item \textbf{Strict:} Il dominio deve corrispondere esattamente
    \item \textbf{Relaxed:} Sottodomini accettati
\end{itemize}

\subsubsection{Email Spoofing}
\textbf{Definizione:} Falsificazione dell'indirizzo mittente di un'email per far sembrare che provenga da una fonte diversa da quella reale.

\textbf{Tecniche comuni:}
\begin{itemize}
    \item Manipolazione header FROM
    \item Uso di domini simili (lookalike domains)
    \item Display name spoofing
    \item Reply-To manipulation
\end{itemize}

\subsubsection{Implementazione nel Sistema}

Il sistema implementa una verifica completa dei protocolli di autenticazione:

\begin{itemize}
    \item \textbf{DKIM Analysis}: Verifica firme digitali e algoritmi crittografici
    \item \textbf{SPF Verification}: Controllo autorizzazioni server mittente
    \item \textbf{DMARC Policy}: Verifica politiche dominio e allineamento
    \item \textbf{Security Assessment}: Valutazione complessiva autenticità email
\end{itemize}

\section{Servizi di Sicurezza Integrati}

\subsection{Servizi Primari}

\subsubsection{VirusTotal}
\textbf{Definizione:} Servizio online gratuito che analizza file e URL utilizzando oltre 70 motori antivirus e servizi di rilevamento malware.

\textbf{Funzionalità nell'implementazione:}
\begin{itemize}
    \item \textbf{Scansione Multi-Engine:} Analisi file, URL, IP con 60+ motori antivirus
    \item \textbf{Implementazione:} Context manager per gestione sessioni
    \item \textbf{Caratteristiche:} Retry logic e cache intelligente
    \item \textbf{Database:} Accesso a storico di minacce conosciute
    \item \textbf{API:} Integrazione automatizzata con limite chiamate
\end{itemize}

\subsubsection{PhishTank}
\textbf{Definizione:} Database collaborativo e gratuito di URL di phishing verificati dalla community, gestito da OpenDNS (Cisco).

\textbf{Implementazione nel sistema:}
\begin{itemize}
    \item \textbf{Funzionalità:} Database phishing community-driven
    \item \textbf{Vantaggi:} Gratuito, aggiornamenti real-time
    \item \textbf{API:} Nessuna chiave richiesta
    \item \textbf{Verifica:} Controllo manuale da parte della community
    \item \textbf{Categorizzazione:} Classificazione per tipo di target
\end{itemize}

\subsection{Servizi Specializzati}

\subsubsection{URLScan.io}
\textbf{Definizione:} Servizio di analisi comportamentale di siti web che esegue una scansione completa della pagina in un ambiente sandbox.

\textbf{Capacità di analisi e implementazione:}
\begin{itemize}
    \item \textbf{Funzionalità}: Analisi comportamentale siti web
    \item \textbf{Caratteristiche}: Screenshot della pagina renderizzata, analisi traffico di rete
    \item \textbf{Rilevamento}: Tecnologie utilizzate e JavaScript malevolo
    \item \textbf{API Key}: Opzionale per funzionalità avanzate
    \item \textbf{Sandbox}: Esecuzione sicura in ambiente isolato
\end{itemize}

\subsubsection{MalwareBazaar}
\textbf{Definizione:} Database pubblico e gratuito di campioni malware mantenuto da abuse.ch, specializzato nella condivisione di hash di file malevoli.

\textbf{Implementazione nel sistema:}
\begin{itemize}
    \item \textbf{Funzionalità}: Database specializzato malware
    \item \textbf{Caratteristiche}: Hash lookup, identificazione famiglia
    \item \textbf{Vantaggi}: Gratuito, database sempre aggiornato
    \item \textbf{Metadati}: Classificazione dettagliata e timeline scoperta
    \item \textbf{API}: Facile integrazione senza limiti stringenti
\end{itemize}

\subsubsection{AbuseIPDB}
\begin{itemize}
    \item \textbf{Funzionalità}: Reputazione IP e report abusi
    \item \textbf{Caratteristiche}: Geolocalizzazione, ISP info, confidence score
    \item \textbf{API}: Chiave richiesta per funzionalità complete
    \item \textbf{Community}: Report dalla community di sicurezza
    \item \textbf{Dati Storici}: Storico delle segnalazioni e attività sospette
\end{itemize}

\subsection{Servizi Aggiuntivi}

Il sistema supporta inoltre:
\begin{itemize}
    \item \textbf{URLVoid}: Aggregazione 30+ motori reputazione
    \item \textbf{Spamhaus}: DNS blacklist per domini
    \item \textbf{Google Safe Browsing}: Protezione Google integrata
\end{itemize}

\section{Sistema di Risk Scoring}

\subsection{Metodologia di Scoring}

Il sistema implementa un algoritmo di risk scoring che combina i risultati di tutti i servizi disponibili:

\begin{itemize}
    \item \textbf{Scala 0-10}: Punteggio rischio standardizzato
    \item \textbf{Peso Bilanciato}: Servizi diversi hanno pesi diversi basati su affidabilita
    \item \textbf{Raccomandazioni Automatiche}: SAFE, WARNING, CAUTION, BLOCK
\end{itemize}

\begin{lstlisting}[language=Python, caption=Esempio algoritmo risk scoring]
def calculate_risk_score(service_results):
    """
    Calcola risk score combinando risultati di servizi multipli
    Scale: 0-10 (0 = sicuro, 10 = massimo rischio)
    """
    total_score = 0
    max_possible_score = 0
    
    # Pesi per servizi URL
    url_weights = {
        'phishtank': 4,      # Alta affidabilita' per phishing
        'virustotal': 3,     # Standard de facto
        'urlvoid': 2,        # Aggregatore utile
        'urlscan': 2,        # Analisi comportamentale
        'google_safe_browsing': 1  # Baseline Google
    }
    
    # Calcola score per servizi URL
    for service, weight in url_weights.items():
        if service in service_results:
            result = service_results[service]
            max_possible_score += weight
            
            if result.get('is_malicious', False):
                total_score += weight
            elif result.get('positives', 0) > 0:
                # Score parziale basato su detection ratio
                ratio = result.get('positives', 0) / max(result.get('total', 1), 1)
                total_score += weight * ratio
    
    # Normalizza su scala 0-10
    if max_possible_score > 0:
        risk_score = min(10, (total_score / max_possible_score) * 10)
    else:
        risk_score = 0
    
    return round(risk_score, 1)

def get_recommendation(risk_score):
    """Fornisce raccomandazione basata su risk score"""
    if risk_score <= 2:
        return "SAFE"
    elif risk_score <= 4:
        return "WARNING"
    elif risk_score <= 7:
        return "CAUTION"
    else:
        return "BLOCK"
\end{lstlisting}

\subsection{Criteri di Valutazione}

\begin{itemize}
    \item \textbf{URL Risk Scoring}: PhishTank (4 punti), VirusTotal (3 punti), URLVoid (2 punti)
    \item \textbf{Hash Risk Scoring}: MalwareBazaar (6 punti), VirusTotal (4 punti)
    \item \textbf{IP Risk Scoring}: AbuseIPDB confidence + VirusTotal engines
\end{itemize}

\section{Ottimizzazioni e Performance}

\subsection{Sistema di Caching}

Il sistema implementa un sofisticato sistema di caching per ottimizzare le performance:

\begin{itemize}
    \item \textbf{File-based Caching}: Risultati API salvati in file JSON
    \item \textbf{Cache Key Generation}: Hash MD5 di URL/IP per identificazione univoca
    \item \textbf{Automatic Expiration}: Scadenza automatica dopo 7 giorni (configurabile)
    \item \textbf{Memory Efficiency}: Deduplicazione in memoria prima delle chiamate API
\end{itemize}

\subsection{Gestione Errori e Resilienza}

\paragraph{Error Handling Comprehensive}

\begin{itemize}
    \item \textbf{Retry Logic}: Exponential backoff per errori transitori
    \item \textbf{Graceful Degradation}: Continuazione analisi anche con servizi non disponibili
    \item \textbf{Quota Management}: Gestione intelligente quote API
    \item \textbf{Fallback System}: Sistema di fallback automatico tra servizi
\end{itemize}

\subsection{Batch Processing}

Per rispettare i rate limits dei servizi esterni:

\begin{itemize}
    \item Elaborazione URL in lotti di 5
    \item Pausa automatica tra batch
    \item Monitoraggio rate limits in real-time
    \item Prioritizzazione richieste critiche
\end{itemize}

\section{Implementazione Tecnica}

\subsection{Requisiti Sistema}

\begin{itemize}
    \item Python 3.8+
    \item Librerie principali:
    \begin{itemize}
        \item \texttt{imaplib}: Connessioni IMAP
        \item \texttt{requests}: Chiamate API HTTP
        \item \texttt{dnspython}: Verifiche DNS per DMARC
        \item \texttt{aiohttp}: Chiamate asincrone (VirusTotal)
        \item \texttt{hashlib}: Calcolo hash file
        \item \texttt{email}: Parsing email RFC-compliant
    \end{itemize}
\end{itemize}

\subsection{Configurazione}

\subsubsection{File di Configurazione}
\begin{lstlisting}[language=Python, caption=Configurazione chiavi API]
# Configurazione API Keys
VIRUSTOTAL_API_KEY = "your_virustotal_api_key_here"
URLSCAN_API_KEY = "your_urlscan_api_key_here"
ABUSEIPDB_API_KEY = "your_abuseipdb_api_key_here"
URLVOID_API_KEY = "your_urlvoid_api_key_here"
GOOGLE_SAFE_BROWSING_API_KEY = "your_google_api_key_here"

# URLs dei servizi
VIRUSTOTAL_BASE_URL = "https://www.virustotal.com/vtapi/v2/"
URLSCAN_BASE_URL = "https://urlscan.io/api/v1/"
PHISHTANK_URL = "http://checkurl.phishtank.com/checkurl/"
MALWAREBAZAAR_URL = "https://mb-api.abuse.ch/api/v1/"

# Configurazione analisi sicurezza
SECURITY_ANALYSIS_CONFIG = {
    'enable_multiple_services': True,
    'virustotal_enabled': True,
    'urlscan_enabled': True,
    'phishtank_enabled': True,
    'malwarebazaar_enabled': True,
    'abuseipdb_enhanced': True,
    'batch_size': 5,
    'request_delay': 1.0,
    'max_retries': 3
}
\end{lstlisting}

\subsubsection{Variabili d'Ambiente}
Il sistema supporta configurazione tramite variabili d'ambiente per maggiore sicurezza:

\begin{lstlisting}[language=bash, caption=Configurazione ambiente]
# Windows PowerShell
$env:VIRUSTOTAL_API_KEY="your_api_key"
$env:URLSCAN_API_KEY="your_api_key"
$env:ABUSEIPDB_API_KEY="your_api_key"

# Linux/Mac
export VIRUSTOTAL_API_KEY="your_api_key"
export URLSCAN_API_KEY="your_api_key"
export ABUSEIPDB_API_KEY="your_api_key"
\end{lstlisting}

\subsection{Utilizzo}

\subsubsection{Interfaccia Command Line}
\begin{lstlisting}[language=bash, caption=Esempi utilizzo CLI]
# Analisi completa con investigazione
python app.py -s imap.server.com -u user@example.com -p password -m INBOX -o emails --complete --investigate

# Analisi file locali
python app.py -f emails/ -i -o results.html

# Solo analisi header
python app.py -f emails/ --header -o headers.json
\end{lstlisting}

\subsubsection{API Programmatica}
\begin{lstlisting}[language=Python, caption=Utilizzo programmatico]
from connectors import comprehensive_url_analysis, comprehensive_hash_analysis

# Analisi URL comprehensive
url_result = comprehensive_url_analysis("https://suspicious-url.com")
print(f"Risk Score: {url_result['risk_score']}/10")
print(f"Recommendation: {url_result['recommendation']}")

# Analisi hash comprehensive  
hash_result = comprehensive_hash_analysis("file_hash_here")
print(f"Malware Family: {hash_result.get('malware_family', 'Unknown')}")
\end{lstlisting}

\section{Output e Visualizzazione}

\subsection{Formato JSON}

Il sistema genera output JSON strutturato con le seguenti sezioni principali:

\begin{itemize}
    \item \textbf{Information}: Metadata scansione (timestamp, file analizzato)
    \item \textbf{Headers}: Analisi completa header con investigazione IP
    \item \textbf{Links}: Analisi URL con risk scoring multi-servizio
    \item \textbf{Digests}: Hash file con verifica malware
    \item \textbf{Authentication}: Risultati DKIM, SPF, DMARC
\end{itemize}

\subsection{Spiegazione Dettagliata Campi JSON}

\subsubsection{Sezione Information}

La sezione \texttt{Information} contiene i metadati della scansione:

\begin{itemize}
    \item \textbf{Filename}: Path completo del file email analizzato
    \item \textbf{Generated}: Timestamp di quando è stata eseguita l'analisi
    \item \textbf{Scan\_Type}: Tipo di scansione (completa, header-only, etc.)
    \item \textbf{Investigation\_Mode}: Indica se è stata eseguita l'investigazione con servizi esterni
\end{itemize}

\subsubsection{Sezione Headers - Data}

Contiene i metadati estratti dall'email:

\begin{itemize}
    \item \textbf{from}: Indirizzo mittente estratto dall'header FROM
    \item \textbf{to}: Destinatario principale dall'header TO
    \item \textbf{subject}: Oggetto dell'email decodificato
    \item \textbf{date}: Data di invio parsata dall'header DATE
    \item \textbf{message-id}: Identificatore univoco dell'email
    \item \textbf{received}: Catena completa header RECEIVED per tracciamento percorso
    \item \textbf{content-type}: Tipo MIME del contenuto email
    \item \textbf{x-spam-status}: Risultato analisi antispam del server
    \item \textbf{dkim-signature}: Firma digitale DKIM se presente
    \item \textbf{authentication-results}: Risultati verifica SPF/DKIM/DMARC
\end{itemize}

\subsubsection{Sezione Headers - Investigation}

Risultati dell'investigazione di sicurezza:

\begin{itemize}
    \item \textbf{X-Sender-Ip}: 
    \begin{itemize}
        \item \texttt{IP}: Indirizzo IP estratto dall'header RECEIVED
        \item \texttt{Virustotal}: Link diretto per verificare IP su VirusTotal
        \item \texttt{Abuseipdb}: Link per controllo reputazione su AbuseIPDB
        \item \texttt{Safety}: Valutazione sicurezza (Safe/Suspicious/Malicious)
        \item \texttt{Positives}: Numero motori antivirus che segnalano l'IP
        \item \texttt{Country}: Geolocalizzazione IP
        \item \texttt{ISP}: Provider internet dell'IP
        \item \texttt{Abuse\_Confidence}: Percentuale confidenza abuso (0-100\%)
    \end{itemize}
    \item \textbf{Blacklist\_Check}:
    \begin{itemize}
        \item \texttt{Blacklist\_Status}: Stato nelle blacklist principali
        \item \texttt{Listed\_In}: Elenco blacklist che contengono l'IP
        \item \texttt{Spamhaus\_SBL}: Presenza in Spamhaus SBL
        \item \texttt{Spamcop}: Presenza in SpamCop blacklist
    \end{itemize}
    \item \textbf{Spoof\_Check}:
    \begin{itemize}
        \item \texttt{Reply-To}: Indirizzo Reply-To se diverso da FROM
        \item \texttt{From}: Indirizzo FROM originale
        \item \texttt{Conclusion}: Valutazione possibile spoofing
    \end{itemize}
\end{itemize}

\subsubsection{Sezione Links - Data}

Elenco numerato di tutti i link estratti:

\begin{itemize}
    \item \textbf{Numerazione}: Ogni link ha un ID numerico progressivo
    \item \textbf{URL}: Link completo estratto dal contenuto email
    \item \textbf{Type}: Tipo di link (HTTP, HTTPS, MAILTO, FTP)
    \item \textbf{Domain}: Dominio estratto dall'URL
    \item \textbf{Path}: Percorso specifico dell'URL
\end{itemize}

\subsubsection{Sezione Links - Investigation}

Analisi di sicurezza per ogni link:

\begin{itemize}
    \item \textbf{Comprehensive\_Analysis}:
    \begin{itemize}
        \item \texttt{Risk\_Score}: Punteggio rischio 0-10 calcolato
        \item \texttt{Recommendation}: Raccomandazione (SAFE/WARNING/CAUTION/BLOCK)
        \item \texttt{Total\_Services}: Numero servizi che hanno analizzato l'URL
        \item \texttt{Malicious\_Count}: Servizi che lo segnalano come malevolo
    \end{itemize}
    \item \textbf{Service\_Results}:
    \begin{itemize}
        \item \texttt{VirusTotal}: Risultati VirusTotal (positives/total, permalink)
        \item \texttt{PhishTank}: Presenza nel database phishing
        \item \texttt{URLScan}: Risultati analisi comportamentale
        \item \texttt{URLVoid}: Aggregazione motori reputazione
        \item \texttt{Google\_Safe\_Browsing}: Risultato Google Safe Browsing
    \end{itemize}
    \item \textbf{Fallback\_Links}:
    \begin{itemize}
        \item \texttt{Virustotal}: Link diretto ricerca manuale
        \item \texttt{Urlscan}: Link ricerca URLScan.io
        \item \texttt{Google}: Link ricerca Google Safe Browsing
    \end{itemize}
\end{itemize}

\subsubsection{Sezione Digests - Data}

Hash calcolati per il file email:

\begin{itemize}
    \item \textbf{File\_MD5}: Hash MD5 del file .eml completo
    \item \textbf{File\_SHA1}: Hash SHA1 del file .eml completo  
    \item \textbf{File\_SHA256}: Hash SHA256 del file .eml completo
    \item \textbf{Content\_MD5}: Hash MD5 del solo contenuto email
    \item \textbf{Content\_SHA1}: Hash SHA1 del solo contenuto email
    \item \textbf{Content\_SHA256}: Hash SHA256 del solo contenuto email
    \item \textbf{Attachment\_Hashes}: Array di hash per ogni allegato
\end{itemize}

\subsubsection{Sezione Digests - Investigation}

Verifica sicurezza degli hash:

\begin{itemize}
    \item \textbf{Per ogni hash}:
    \begin{itemize}
        \item \texttt{Virustotal}: Link diretto controllo hash su VirusTotal
        \item \texttt{MalwareBazaar}: Risultato controllo su MalwareBazaar
        \item \texttt{Is\_Malware}: Boolean indicante se riconosciuto come malware
        \item \texttt{Malware\_Family}: Famiglia malware identificata
        \item \texttt{First\_Seen}: Data prima identificazione
        \item \texttt{Detection\_Ratio}: Rapporto detection motori antivirus
    \end{itemize}
\end{itemize}

\subsubsection{Sezione Authentication (DMARC/DKIM)}

Risultati verifica protocolli autenticazione:

\begin{itemize}
    \item \textbf{DMARC\_Policy}:
    \begin{itemize}
        \item \texttt{Policy}: Politica DMARC del dominio (none/quarantine/reject)
        \item \texttt{Percentage}: Percentuale email sottoposte a policy
        \item \texttt{Subdomain\_Policy}: Policy per sottodomini
        \item \texttt{DKIM\_Alignment}: Modalità allineamento DKIM
        \item \texttt{SPF\_Alignment}: Modalità allineamento SPF
    \end{itemize}
    \item \textbf{DKIM\_Analysis}:
    \begin{itemize}
        \item \texttt{Signature\_Valid}: Validità strutturale firma DKIM
        \item \texttt{Algorithm}: Algoritmo crittografico utilizzato
        \item \texttt{Domain}: Dominio che ha firmato l'email
        \item \texttt{Selector}: Selettore DKIM utilizzato
        \item \texttt{Headers\_Signed}: Lista header protetti dalla firma
        \item \texttt{Body\_Hash}: Hash del corpo email nella firma
    \end{itemize}
    \item \textbf{SPF\_Results}:
    \begin{itemize}
        \item \texttt{Result}: Risultato verifica SPF (pass/fail/softfail/neutral)
        \item \texttt{IP\_Authorized}: Se l'IP mittente è autorizzato dal record SPF
        \item \texttt{SPF\_Record}: Record SPF del dominio mittente
    \end{itemize}
\end{itemize}

\subsubsection{Sezione Risk\_Assessment}

Valutazione complessiva del rischio:

\begin{itemize}
    \item \textbf{Overall\_Risk\_Score}: Punteggio rischio complessivo 0-10
    \item \textbf{Risk\_Factors}: Array dei fattori di rischio identificati
    \item \textbf{Recommendation}: Raccomandazione finale di sicurezza
    \item \textbf{Confidence\_Level}: Livello confidenza nell'analisi (0-100\%)
    \item \textbf{Categories}: Categorie di minacce identificate (phishing, malware, spam)
\end{itemize}

\subsection{Report HTML Interattivo}

\paragraph{Interfaccia Web Avanzata}

Il sistema genera report HTML con:

\begin{itemize}
    \item \textbf{Dashboard Overview}: Riepilogo rischi e raccomandazioni
    \item \textbf{Sezioni Expandibili}: Contenuto organizzato in sezioni pieghevoli
    \item \textbf{Color Coding}: Codifica colori per livelli di rischio
    \item \textbf{Interactive Elements}: JavaScript per navigazione migliorata
    \item \textbf{Responsive Design}: Compatibilità dispositivi mobili
\end{itemize}

\subsection{Esempio Output JSON Completo con Spiegazioni}

Di seguito un esempio completo di output JSON con commenti esplicativi:

\begin{lstlisting}[caption=Esempio output JSON dettagliato]
{
    "Information": {
        "Scan": {
            "Filename": "emails/phishing_attempt.eml",
            "Generated": "June 17, 2025 - 14:30:22",
            "Scan_Type": "complete_investigation",
            "Processing_Time": "45.3 seconds",
            "Services_Used": 8
        }
    },
    "Analysis": {
        "Headers": {
            "Data": {
                "from": "security@bank-example.com",
                "to": "victim@company.com", 
                "subject": "=?UTF-8?B?VXJnZW50IEFjY291bnQgVmVyaWZpY2F0aW9u?=",
                "date": "Mon, 17 Jun 2025 12:15:30 +0200",
                "message-id": "<suspicious123@attacker-server.com>",
                "received": "from mail.attacker.com ([203.0.113.45]) by victim-server.com",
                "x-spam-status": "Yes, score=8.5 required=5.0",
                "authentication-results": "victim-server.com; dkim=fail spf=fail dmarc=fail"
            },
            "Investigation": {
                "X-Sender-Ip": {
                    "IP": "203.0.113.45",
                    "Virustotal": "https://www.virustotal.com/gui/search/203.0.113.45",
                    "Abuseipdb": "https://www.abuseipdb.com/check/203.0.113.45",
                    "Safety": "Malicious",
                    "Positives": 12,
                    "Total_Engines": 67,
                    "Country": "Unknown",
                    "ISP": "Bulletproof Hosting",
                    "Abuse_Confidence": 89,
                    "Last_Seen": "2025-06-16"
                },
                "Blacklist_Check": {
                    "Blacklist_Status": "Listed in 3 blacklists",
                    "Spamhaus_SBL": "Listed",
                    "Spamcop": "Listed", 
                    "SURBL": "Listed",
                    "Details": "IP known for hosting phishing campaigns"
                },
                "Authentication_Summary": {
                    "SPF_Result": "fail",
                    "DKIM_Result": "fail", 
                    "DMARC_Result": "fail",
                    "Overall_Auth": "FAILED - High spoofing risk"
                }
            }
        },
        "Links": {
            "Data": {
                "1": "https://secure-bank-login.evil.com/verify",
                "2": "https://fonts.googleapis.com/css2?family=Roboto",
                "3": "mailto:support@bank-example.com"
            },
            "Investigation": {
                "1": {
                    "Comprehensive_Analysis": {
                        "Risk_Score": 9.2,
                        "Recommendation": "BLOCK",
                        "Total_Services": 5,
                        "Malicious_Count": 4,
                        "Confidence": "Very High"
                    },
                    "Service_Results": {
                        "VirusTotal": {
                            "Positives": 45,
                            "Total": 67,
                            "Categories": ["phishing", "malware"],
                            "Last_Analysis": "2025-06-17T12:30:15Z"
                        },
                        "PhishTank": {
                            "In_Database": true,
                            "Verified": true,
                            "Submission_Date": "2025-06-15",
                            "Target": "Banking Phishing"
                        },
                        "URLScan": {
                            "Verdict": "malicious",
                            "Screenshot_Available": true,
                            "Technologies": ["PHP", "JavaScript obfuscation"],
                            "Final_URL": "https://evil.com/steal-credentials.php"
                        },
                        "URLVoid": {
                            "Detections": 23,
                            "Total_Engines": 30,
                            "Risk_Score": 87
                        }
                    }
                },
                "2": {
                    "Comprehensive_Analysis": {
                        "Risk_Score": 0.1,
                        "Recommendation": "SAFE",
                        "Notes": "Legitimate Google Fonts service"
                    }
                }
            }
        },
        "Digests": {
            "Data": {
                "File_MD5": "a1b2c3d4e5f6g7h8i9j0k1l2m3n4o5p6",
                "File_SHA1": "da39a3ee5e6b4b0d3255bfef95601890afd80709", 
                "File_SHA256": "e3b0c44298fc1c149afbf4c8996fb92427ae41e4649b934ca495991b7852b855",
                "Content_MD5": "f6e7d8c9b0a1s2d3f4g5h6j7k8l9m0n1",
                "Attachment_Count": 1,
                "Attachment_Hashes": [
                    {
                        "Filename": "invoice.pdf.exe",
                        "MD5": "malware123456789abcdef",
                        "SHA256": "suspicious_file_hash_here",
                        "Size": 245760
                    }
                ]
            },
            "Investigation": {
                "File_SHA256": {
                    "Virustotal": "https://www.virustotal.com/gui/search/e3b0c44...",
                    "Is_Known": false,
                    "Reputation": "Unknown"
                },
                "Attachment_Analysis": {
                    "invoice.pdf.exe": {
                        "MalwareBazaar": {
                            "Is_Malware": true,
                            "Malware_Family": "AgentTesla",
                            "First_Seen": "2025-06-10",
                            "Confidence": "High"
                        },
                        "VirusTotal": {
                            "Detections": 56,
                            "Total_Engines": 67,
                            "Categories": ["trojan", "stealer"],
                            "Recommendation": "QUARANTINE"
                        }
                    }
                }
            }
        },
        "Authentication": {
            "DMARC_Analysis": {
                "Domain": "bank-example.com",
                "Policy": "reject",
                "Percentage": 100,
                "Alignment_SPF": "strict",
                "Alignment_DKIM": "strict",
                "Report_URI": "mailto:dmarc@bank-example.com",
                "Evaluation": "Policy violation - should be rejected"
            },
            "DKIM_Analysis": {
                "Signature_Present": false,
                "Expected_Domain": "bank-example.com",
                "Evaluation": "FAIL - No valid DKIM signature found",
                "Risk_Level": "High - Possible domain spoofing"
            },
            "SPF_Analysis": {
                "Result": "fail",
                "Sender_IP": "203.0.113.45",
                "SPF_Record": "v=spf1 include:_spf.bank-example.com -all",
                "Evaluation": "IP not authorized to send for this domain"
            }
        },
        "Risk_Assessment": {
            "Overall_Risk_Score": 9.5,
            "Risk_Level": "CRITICAL",
            "Primary_Threats": [
                "Phishing attempt",
                "Malware delivery", 
                "Domain spoofing",
                "Credential theft"
            ],
            "Recommendation": "BLOCK AND QUARANTINE",
            "Confidence": 98,
            "Action_Required": "Immediate security team notification"
        }
    }
}
\end{lstlisting}


\subsection{Interpretazione dei Valori di Risk Score}

Il sistema utilizza una scala standardizzata per il risk scoring:

\begin{itemize}
    \item \textbf{0.0 - 2.0}: \textcolor{green}{SAFE} - Nessun rischio identificato
    \item \textbf{2.1 - 4.0}: \textcolor{orange}{WARNING} - Rischio basso, monitoraggio consigliato  
    \item \textbf{4.1 - 7.0}: \textcolor{red}{CAUTION} - Rischio moderato, analisi approfondita necessaria
    \item \textbf{7.1 - 10.0}: \textcolor{red}{\textbf{BLOCK}} - Rischio elevato, azione immediata richiesta
\end{itemize}

\subsection{Significato dei Campi di Autenticazione}

\begin{itemize}
    \item \textbf{SPF (Sender Policy Framework)}: Verifica che l'IP mittente sia autorizzato dal dominio
    \item \textbf{DKIM (DomainKeys Identified Mail)}: Verifica firma crittografica del dominio mittente
    \item \textbf{DMARC (Domain-based Message Authentication)}: Policy che combina SPF e DKIM
    \item \textbf{Authentication-Results}: Header che riassume i risultati di tutte le verifiche
\end{itemize}

\section{Testing e Validazione}

\subsection{Test Suite Implementata}

Il sistema è stato validato attraverso:

\begin{itemize}
    \item \textbf{Unit Tests}: Test individuali per ogni modulo analyzer
    \item \textbf{Integration Tests}: Test end-to-end con email reali
    \item \textbf{Performance Tests}: Verifica performance con volumi elevati
    \item \textbf{Security Tests}: Validazione gestione errori e input malformati
\end{itemize}

\subsection{Risultati Testing}

\begin{itemize}
    \item  \textbf{4 email di test}: Analizzate con successo senza errori
    \item  \textbf{Tutti i moduli}: Import e funzionamento corretto
    \item  \textbf{Output generation}: JSON e HTML generati correttamente
    \item  \textbf{Error handling}: Gestione graceful di errori API
    \item  \textbf{Performance}: Riduzione 70\% chiamate API grazie al caching
\end{itemize}

\section{Vantaggi e Benefici}

\subsection{Riduzione Dipendenza Singolo Fornitore}

\begin{itemize}
    \item \textbf{Prima}: 100\% dipendenza da VirusTotal
    \item \textbf{Dopo}: Sistema distribuito su 8+ servizi
    \item \textbf{Beneficio}: Resilienza e coverage migliorata
\end{itemize}

\subsection{Miglioramento Detection Rate}

\begin{itemize}
    \item \textbf{Phishing}: PhishTank specializzato + VirusTotal
    \item \textbf{Malware}: MalwareBazaar + VirusTotal per coverage completa
    \item \textbf{IP Reputation}: AbuseIPDB + blacklist multiple
    \item \textbf{Website Analysis}: URLScan.io per analisi comportamentale
\end{itemize}

\subsection{Performance e Affidabilità}

\begin{itemize}
    \item \textbf{Context Managers}: Eliminazione memory leaks
    \item \textbf{Batch Processing}: Rispetto rate limits automatico
    \item \textbf{Intelligent Caching}: Riduzione chiamate API duplicate
    \item \textbf{Error Handling}: Graceful degradation quando servizi non disponibili
\end{itemize}

\section{Sicurezza e Compliance}

\subsection{Gestione Credenziali}

\begin{itemize}
    \item Chiavi API memorizzate in variabili d'ambiente
    \item Nessuna chiave hardcoded nel codice sorgente
    \item Supporto per sistemi di gestione credenziali enterprise
\end{itemize}

\subsection{Privacy e GDPR}

\begin{itemize}
    \item Analisi locale senza upload email complete
    \item Solo hash e URL condivisi con servizi esterni
    \item Cache locale con scadenza automatica
    \item Possibilità di disabilitare servizi specifici
\end{itemize}

\section{Casi d'Uso}

\subsection{SOC e Security Teams}

\begin{itemize}
    \item Analisi incident response per email sospette
    \item Verifica campagne phishing
    \item Intelligence gathering su attaccanti
    \item Validazione sistemi email aziendali
\end{itemize}

\subsection{Amministratori Email}

\begin{itemize}
    \item Audit configurazioni DMARC/DKIM/SPF
    \item Monitoraggio reputazione dominio
    \item Verifica efficacia filtri antispam
    \item Analisi deliverability email
\end{itemize}

\subsection{Ricerca e Formazione}

\begin{itemize}
    \item Studio tecniche phishing avanzate
    \item Analisi evoluzione malware email-based
    \item Training su protocolli autenticazione email
    \item Benchmark sistemi di security
\end{itemize}

\section{Limitazioni e Considerazioni}

\subsection{Limitazioni Tecniche}

\begin{itemize}
    \item \textbf{Rate Limits}: Dipendenti dai limiti API dei servizi esterni
    \item \textbf{Latenza}: Analisi comprehensive richiede più tempo
    \item \textbf{Dipendenza Rete}: Richiede connessione internet per investigation
    \item \textbf{False Positives}: Possibili con servizi multipli
\end{itemize}

\subsection{Considerazioni Economiche}

\begin{itemize}
    \item Alcuni servizi richiedono subscription a pagamento
    \item Costi API possono scalare con volume analisi
    \item Necessità bilanciamento costo/beneficio per deployments large-scale
\end{itemize}

\section{Roadmap e Sviluppi Futuri}

\subsection{Miglioramenti Immediati}

\begin{itemize}
    \item \textbf{Machine Learning}: Integrazione ML per pattern recognition
    \item \textbf{Real-time Processing}: Supporto analisi email in tempo reale
    \item \textbf{API REST}: Esposizione funzionalità via API REST
    \item \textbf{Dashboard Web}: Interfaccia web per gestione e monitoring
\end{itemize}

\subsection{Integrazioni Avanzate}

\begin{itemize}
    \item \textbf{SIEM Integration}: Connettori per Splunk, ELK Stack
    \item \textbf{Threat Intelligence}: Integrazione feed intelligence commerciali
    \item \textbf{Automated Response}: Azioni automatiche basate su risk score
    \item \textbf{Blockchain Verification}: Verifica autenticità tramite blockchain
\end{itemize}

\subsection{Integrazioni Avanzate}

\begin{itemize}
    \item \textbf{SIEM Integration}: Connettori per Splunk, ELK Stack
    \item \textbf{Threat Intelligence}: Integrazione feed intelligence commerciali
    \item \textbf{Automated Response}: Azioni automatiche basate su risk score
    \item \textbf{Blockchain Verification}: Verifica autenticità tramite blockchain
\end{itemize}

\section{Conclusioni}

L'Email Analyzer rappresenta una soluzione completa e moderna per l'analisi di sicurezza delle email, che evolve da un sistema mono-servizio a una piattaforma multi-servizio robusta e scalabile.

\subsection{Obiettivi Raggiunti}

\begin{itemize}
    \item  \textbf{Diversificazione Servizi}: Ridotta dipendenza da singolo fornitore
    \item  \textbf{Miglioramento Coverage}: Incremento significativo detection rate
    \item  \textbf{Ottimizzazione Performance}: Sistema di caching e batch processing
    \item  \textbf{Resilienza}: Graceful degradation e error handling avanzato
    \item  \textbf{Usabilità}: Output HTML interattivo e CLI user-friendly
\end{itemize}

\subsection{Valore Aggiunto}

Il sistema fornisce valore significativo attraverso:

\begin{itemize}
    \item \textbf{Accuratezza}: Analisi multi-servizio riduce false negative
    \item \textbf{Completezza}: Coverage completa di vettori di attacco email
    \item \textbf{Efficienza}: Automazione completa processo di analisi
    \item \textbf{Scalabilità}: Architettura modulare facilmente estendibile
    \item \textbf{Affidabilità}: Sistema robusto con fallback multipli
\end{itemize}

Il progetto dimostra come un'architettura modulare ben progettata possa evolvere da un prototipo semplice a una soluzione enterprise-grade, mantenendo facilità d'uso e affidabilità operativa.


\end{document}
